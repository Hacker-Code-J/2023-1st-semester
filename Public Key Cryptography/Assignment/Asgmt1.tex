\documentclass{article}

% Packages for formatting
\usepackage[margin=1in]{geometry}
\usepackage{fancyhdr}
\usepackage{enumitem}
\usepackage{graphicx}
\usepackage{kotex}
\usepackage{amsmath}
\usepackage{amsthm}
\usepackage{algorithm2e,setspace}
\usepackage{algpseudocode}
\usepackage{xcolor}
\usepackage{amssymb}

% Define colors
\definecolor{blue1}{HTML}{0077c2}
\definecolor{blue2}{HTML}{00a5e6}
\definecolor{blue3}{HTML}{b3e0ff}
\definecolor{blue4}{HTML}{00293c}
\definecolor{blue5}{HTML}{e6f7ff}
\usepackage{color,soul}
\usepackage{soul}
\newcommand{\mathcolorbox}[2]{\colorbox{#1}{$\displaystyle #2$}}
%Tcolorbox
\usepackage[most]{tcolorbox}
\tcbset{enhanced, colback=white,colframe=black,fonttitle=\bfseries,arc=4mm,boxrule=1pt,shadow={2mm}{-1mm}{0mm}{black!50}}

% Header and footer formatting
\pagestyle{fancy}
\fancyhf{}
\rhead{Student ID: 20192250\quad Name: 지용현}%\rule{3cm}{0.4pt}}
\lhead{\textcolor{blue2}{\textbf{PKC Assignment \#1}}}
% Define footer
\newcommand{\footer}[1]{
	\begin{flushright}
		\vspace{2em}
		\includegraphics[width=2cm]{school_logo.jpg} \\
		\vspace{1em}
		\textcolor{blue2}{\small\textbf{#1}}
	\end{flushright}
}
%\rfoot{\large Department of Information Security, Cryptogrphy and Mathematics, Kookmin Uni.\includegraphics[height=1.5cm]{school_logo.jpg}}

\newtheorem{theorem}{Theorem}
\newtheorem*{theorem*}{Theorem}

\newcommand{\ie}{\textnormal{i.e.}}
\newcommand{\rsa}{\mathsf{RSA}}
\newcommand{\rsacrt}{\mathsf{RSA}\textendash\mathsf{CRT}}
\newcommand{\inv}[1]{#1^{-1}}

\begin{document}

\begin{center}
	$\underset{\LARGE\textbf{(Public Key Cryptography)}}{\huge\textbf{공개키 암호}}$
\end{center}
\begin{itemize}
	\item[\normalfont\bf\textcolor{magenta}{Q}.] $\rsa$의 개인키의 크기를 줄이시오. \begin{proof}[\normalfont\bf\textcolor{magenta}{Answer}]
		Recall that Chinese Remainder Theorem ($\mathsf{CRT}$) : 
		\iffalse
		\begin{tcolorbox}[title=Chinese Remainder Theorem (CRT) - Special Case]
			\begin{theorem*}
				Given a system of $k$ linear congruences:
				\begin{align*}
					x&\equiv a_1 \pmod{m_1}\\
					x&\equiv a_2 \pmod{m_2}\\
					&\vdots \\
					x&\equiv a_k \pmod{m_k}
				\end{align*} where $m_1,m_2,\dots, m_k$ are pairwise coprime. Let $M=\prod_{i=1}^km_i$. Then, the unique solution of the system of congruences is give by \begin{align*}
					x&=\sum_{i=1}^ka_iM_ib_i\pmod{M}\\
					&=a_1M_1b_1+a_2M_2b_2+\cdots+a_kM_kb_k\pmod{M}.
				\end{align*} where $M_i=M/m_i$ and $ b_i\equiv\inv{M_i}\pmod{m_i}$.
			\end{theorem*}
		\end{tcolorbox}\fi
		\begin{tcolorbox}[title=Chinese Remainder Theorem (CRT) - Special Case]
		\begin{theorem*}
			Consider a system of two linear congruences:
			\begin{align*}
				x&\equiv a_1 \pmod{p}\\
				x&\equiv a_2 \pmod{q}
			\end{align*} where $p,q$ are coprime. Let $N=pq$. Then, the unique solution of the system of congruences is give by \begin{align*}
				\mathcolorbox{-blue}{x=a_1qq_p^{-1}+a_2pp_q^{-1}\mod{N}}
			\end{align*} where $q_p^{-1}=\inv{q}\mod{p}$ and $p_{q}^{-1}=\inv{p}\mod{q}$.
		\end{theorem*}
		\tcblower
		Recall that Bézout's identity : $
		a,b\in\mathbb{Z}\implies\exists x,y\in\mathbb{Z}:\gcd(a,b)=ax+by.
		$ Especially, \[
		\text{$p,q$ are coprime}\implies\exists x,y\in\mathbb{Z}: px+qy=1.
		\] Let $p,q$ are coprime. Then $\exists x,y\in\mathbb{Z}:px+qy=1$ and so \[
		px=(-y)q+1\quad\leadsto\quad px\equiv1\pmod{q}\quad\leadsto\quad x=p^{-1}\mod{q}.
		\] Similarly, $y=q^{-1}\mod{p}$. Thus we have $px+qy=1\leadsto pp_{q}^{-1}+qq_{p}^{-1}=1$. Consequently, \begin{align*}
			\mathcolorbox{-blue}{x=a_1qq_p^{-1}+a_2pp_q^{-1}\mod{N}} &\leftrightsquigarrow x=a_1qq_p^{-1}+a_2(1-qq_{p}^{-1})\mod{N}\\
			&\leftrightsquigarrow \mathcolorbox{-red}{x=(a_1-a_2)qq_p^{-1}+a_2\mod{N}}\\
		\end{align*}
	\end{tcolorbox}
	continue on next page.
		\newpage
		and that $\rsa$-$\mathsf{CRT}$ algorithm : \\
		\RestyleAlgo{ruled}
		\SetKwComment{Comment}{/* }{ */}
		\begin{algorithm}[H]
			\caption{$\mathsf{RSA}$-$\mathsf{CRT}$ Algorithm}
			\KwData{The security parameter $k$, a public key $(N,e)$, a ciphertext $\mathcal{C}$.}
			\KwResult{The plaintext message $\mathcal{M}$ corresponding to the ciphertext $\mathcal{C}$.}
			\vspace{4pt}
			\vspace{4pt}
			\SetKwFunction{KeyGen}{KeyGen}
			\SetKwProg{Fn}{Function}{:}{}
			\Comment{Key Generation}
			\Fn{\KeyGen$(k)$}{
				$p, q \gets$ random prime numbers of $k/2$ bits each \tcp*{Generate two primes}
				$N \gets pq$ \tcp*{Compute modulus}
				$\phi(N) \gets (p-1)(q-1)$ \tcp*{Compute Euler's phi function}
				$e \gets$ integer s.t. $1 < e < \phi(n)\land \gcd(e, \phi(n)) = 1$ \tcp*{Choose encryption exponent}
				$\mathcolorbox{-green}{d_p\leftarrow e^{-1} \mod p-1}$ \tcp*{Compute decryption exponent for $p$}
				$\mathcolorbox{-green}{d_q \leftarrow e^{-1} \mod q-1}$ \tcp*{Compute decryption exponent for $q$}
				$q_{inv} \gets q^{-1} \mod p$ \tcp*{Compute $q$ inverse modulo $p$}
				Set the $\rsa$ public key as $(N,e)$\;
				Set the $\rsa$ secret key as $(p,q,d_p,d_q,q_{inv})$\;
				\iffalse
				$N \gets \text{length of } A$;
				$low \gets 0$;
				$high \gets n-1$;
				\While{$low \leq high$}{
					$mid \gets \lfloor (low + high) / 2 \rfloor$;
					\If{$A[mid] < x$}{
						$low \gets mid + 1$;
					} \ElseIf{$A[mid] > x$}{
						$high \gets mid - 1$;
					} \Else{
						\KwRet mid;
					}
				}\fi
			}
			\textbf{End Function}\\
			\vspace{4pt}
			\vspace{4pt}
			\Comment{Encryption}
			\SetKwFunction{Enc}{Enc}
			\Fn{\Enc$(N,e,\mathcal{M})$}{
				$\mathcal{C}\gets\mathcal{M}^e\mod N$ \tcp*{Encrypt with $e$ and $N$}
			}
			\textbf{End Function}\\
			\vspace{4pt}
			\vspace{4pt}
			\Comment{Decryption}
			\SetKwFunction{Dec}{Dec}
			\Fn{\Dec$(\mathcal{C})$}{
				$m_1 \gets \mathcal{C}^{d_p} \mod p$ \tcp*{Decrypt with $d_p$ and $p$}
				$m_2 \gets \mathcal{C}^{d_q} \mod q$ \tcp*{Decrypt with $d_q$ and $q$}
				$t \gets q_{inv}(m_1 - m_2)$ \tcp*{Reconstruct the message using CRT}
				$m \leftarrow m_2 + qt\mod N$ \tcp*{$\mathcolorbox{-red}{m=(m_1-m_2)qq_{inv}+m_2\mod N}$}
			}
			\textbf{End Function}\\
			\vspace{2.5pt}
			\KwRet{$m$}\;
		\end{algorithm}
	
		위 알고리즘을 통해 $d_p$, $d_q$, $p$ 및 $q$를 사용하여 공개 키 $(N,e)$로 암호화된 메시지를 복호화할 수 있다는 걸 알 수 있습니다. $d_p$와 $d_q$는 전체 $d$보다 훨씬 작기 때문에 $\rsa$ 개인 키의 길이가 줄어듭니다.
		
		따라서 $\rsa$-$\mathsf{CRT}$는 복호화 지수 $d$를 두 부분으로 분할하여 $\rsa$ 개인 키의 길이를 줄이며 $\rsa$의 복호화 지수를 더 쉽게 계산할 수 있도록 합니다.
	\end{proof}
\end{itemize}

	\footer{Department of Information Security, Cryptography and Mathematics, Kookmin University}
\end{document}
