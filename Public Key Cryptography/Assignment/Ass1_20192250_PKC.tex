\documentclass{article}

% Packages for formatting
\usepackage[margin=1in]{geometry}
\usepackage{fancyhdr}
\usepackage{enumitem}
\usepackage{graphicx}
\usepackage{kotex}
\usepackage{amsmath}
\usepackage{amsthm}
\usepackage{algorithm}

%Tcolorbox
\usepackage[most]{tcolorbox}
\tcbset{enhanced, colback=white,colframe=black,fonttitle=\bfseries,arc=4mm,boxrule=1pt,shadow={2mm}{-1mm}{0mm}{black!50}}

% Header and footer formatting
\pagestyle{fancy}
\fancyhf{}
\rhead{Name: 20192250 지용현}%\rule{3cm}{0.4pt}}
\lhead{\textbf{PKC Assignment \#1}}
\rfoot{\large Department of Information Security, Cryptogrphy and Mathematics, Kookmin Uni.\includegraphics[height=1.5cm]{school_logo.jpg}}

\newtheorem{theorem}{Theorem}
\newtheorem*{theorem*}{Theorem}

\newcommand{\ie}{\textnormal{i.e.}}
\newcommand{\rsa}{\mathsf{RSA}}
\newcommand{\rsacrt}{\mathsf{RSA}\textendash\mathsf{CRT}}
\newcommand{\inv}[1]{#1^{-1}}

\begin{document}
	
	%\begin{center}
	%	$\underset{\LARGE\textbf{(Public Key Cryptography)}}{\huge\textbf{공개키 암호}}$
	%\end{center}
	
	\begin{center}
		\huge Reducing private key space of $\rsa$ with $\mathsf{CRT}$.
	\end{center}
	
	\begin{itemize}
		\item RSA의 개인키의 크기를 줄이시오. \begin{proof}[\normalfont\bf\textcolor{magenta}{[답변]}]
			Recall that Chinese Remainder Theorem ($\mathsf{CRT}$) : 
			\begin{tcolorbox}[title=Chinese Remainder Theorem (CRT)]
				\begin{theorem*}
					Given a system of $k$ linear congruences:
					\begin{align*}
					x&\equiv a_1 \pmod{m_1}\\
					x&\equiv a_2 \pmod{m_2}\\
					&\vdots \\
					x&\equiv a_k \pmod{m_k}
					\end{align*} where $m_1,m_2,\dots, m_k$ are pairwise coprime. Let $M=\prod_{i=1}^km_i$. Then, the unique solution of the system of congruences is give by \begin{align*}
					x&=\sum_{i=1}^ka_iM_ib_i\pmod{M}\\
					&=a_1M_1b_1+a_2M_2b_2+\cdots+a_kM_kb_k\pmod{M}.
					\end{align*} where $M_i=M/m_i$ and $ b_i\equiv\inv{M_i}\pmod{m_i}$.
				\end{theorem*}
			\end{tcolorbox}
			
			and that $\rsacrt$ algorithm :  \iffalse\begin{algorithm}[H]
				\SetAlgoLined
				\KwIn{A public key $(n,e)$, a ciphertext $c$, and the primes $p$ and $q$ such that $n = pq$.}
				\KwOut{The plaintext message corresponding to the ciphertext.}
				$d_p \leftarrow e^{-1} \bmod{(p-1)}$;
				$d_q \leftarrow e^{-1} \bmod{(q-1)}$;
				$q_{inv} \leftarrow q^{-1} \bmod{p}$;
				$m_1 \leftarrow c^{d_p} \bmod{p}$;
				$m_2 \leftarrow c^{d_q} \bmod{q}$;
				$h \leftarrow (q_{inv}(m_1 - m_2) \bmod{p})$;
				$m \leftarrow m_2 + qh$;
				\Return{$m$}
				\caption{RSA-CRT Decryption Algorithm}
			\end{algorithm}
			\fi
			\newpage
			$\rsacrt$However, computing the modular exponentiation $c^d mod n$ can be computationally expensive, especially for large values of d. RSA-CRT provides an optimization that can speed up this process.
			
			The CRT states that if we know the values of a number x modulo two relatively prime numbers p and q, we can compute the value of x modulo pq using the Chinese Remainder Theorem. In the context of RSA-CRT, we can use this property to break down the computation of the modular exponentiation $c^d mod n$ into several smaller operations, as follows:
			
			Compute c mod p and c mod q.
			Compute dp = d mod (p-1) and dq = d mod (q-1).
			Compute $xp = c^dp mod p and xq = c^dq mod q$.
			Use the CRT to combine xp and xq to obtain x mod pq.
			x is the decrypted message m.
			This technique can significantly speed up the decryption process, as it reduces the size of the exponents that need to be computed. However, it also introduces a vulnerability: if an attacker can obtain the values of xp and xq, they can use the CRT to obtain the value of x mod pq, which is equivalent to the decrypted message m. If the values of xp and xq are obtained, an attacker can compute the private exponent d as follows:
			
			$d = (xp - xq) * (q^-1 mod p) mod p$
			
			or
			
			$d = (xq - xp) * (p^-1 mod q) mod q$
			
			If p and q are not large enough, an attacker can factor n and obtain the private key. This means that the effective size of the private key space in RSA-CRT is reduced, as an attacker only needs to factor p or q, rather than n, to obtain the private key.
			
			To mitigate this vulnerability, it is important to choose large prime numbers for p and q, and to use secure methods for generating and storing the private key. Additionally, it may be advisable to use other cryptographic techniques, such as padding and message authentication codes, to further enhance the security of the RSA-CRT system.
		\end{proof}
	\end{itemize}
	
\end{document}
