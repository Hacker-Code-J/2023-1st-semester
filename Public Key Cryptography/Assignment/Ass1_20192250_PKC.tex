\documentclass{article}

% Packages for formatting
\usepackage[margin=1in]{geometry}
\usepackage{fancyhdr}
\usepackage{enumitem}
\usepackage{graphicx}
\usepackage{kotex}
\usepackage{amsmath}
\usepackage{amsthm}
\usepackage{algorithm}

% Header and footer formatting
\pagestyle{fancy}
\fancyhf{}
\rhead{Name: 20192250 지용현}%\rule{3cm}{0.4pt}}
\lhead{\textbf{Assignment \#1}}
\rfoot{\large Department of Information Security, Cryptogrphy and Mathematics, Kookmin Uni.\includegraphics[height=1.5cm]{school_logo.jpg}}

\begin{document}

\begin{center}
    $\underset{\LARGE\textbf{(Public Key Cryptography)}}{\huge\textbf{공개키 암호}}$
\end{center}

\begin{itemize}
	\item RSA의 개인키의 크기를 줄이시오. \begin{proof}
		Recall that RSA-CRT algorithm :  \iffalse\begin{algorithm}[H]
			\SetAlgoLined
			\KwIn{A public key $(n,e)$, a ciphertext $c$, and the primes $p$ and $q$ such that $n = pq$.}
			\KwOut{The plaintext message corresponding to the ciphertext.}
			$d_p \leftarrow e^{-1} \bmod{(p-1)}$;
			$d_q \leftarrow e^{-1} \bmod{(q-1)}$;
			$q_{inv} \leftarrow q^{-1} \bmod{p}$;
			$m_1 \leftarrow c^{d_p} \bmod{p}$;
			$m_2 \leftarrow c^{d_q} \bmod{q}$;
			$h \leftarrow (q_{inv}(m_1 - m_2) \bmod{p})$;
			$m \leftarrow m_2 + qh$;
			\Return{$m$}
			\caption{RSA-CRT Decryption Algorithm}
		\end{algorithm}
		\fi
		Suppose we have a system of $k$ linear congruences:
		
		\begin{align*}
		x &\equiv a_1 \pmod{n_1} \
		x &\equiv a_2 \pmod{n_2} \
		&\vdots \
		x &\equiv a_k \pmod{n_k}
		\end{align*}
		
		where $n_1, n_2, \dots, n_k$ are pairwise coprime. Let $N = n_1 n_2 \cdots n_k$. Then, for each $i = 1, 2, \dots, k$, let $N_i = N/n_i$ and let $d_i$ be the inverse of $N_i$ modulo $n_i$, i.e., $d_iN_i \equiv 1 \pmod{n_i}$. Then the unique solution of the system of congruences is given by:
		
		$$x = \sum_{i=1}^{k} a_i N_i d_i \pmod{N}$$
		
		To prove this, first note that for each $i$, the condition $n_i \mid (x-a_i)$ implies that $n_i \mid (x-a_j)$ for all $j \neq i$. Therefore, if $y$ is any integer that satisfies the system of congruences, then we have $y \equiv x \pmod{n_i}$ for all $i$. In particular, $y-x$ is divisible by each $n_i$, so $N \mid (y-x)$. Thus, any two solutions of the system of congruences differ by a multiple of $N$.
		
		Now, we need to show that $x$ is a solution to the system of congruences. For each $i$, we have:
		
		\begin{align*}
		x &\equiv \sum_{j=1}^{k} a_j N_j d_j \pmod{n_i} \
		&= a_i N_i d_i + \sum_{j \neq i} a_j N_j d_j \pmod{n_i} \
		&\equiv a_i N_i d_i \pmod{n_i} \
		\end{align*}
		
		since $n_i$ divides $N_j$ for all $j \neq i$. Thus, $x$ satisfies the $i$-th congruence. Therefore, $x$ is a solution to the system of congruences.
		
		Finally, we need to show that $x$ is the unique solution modulo $N$. Suppose $y$ is another solution. Then $y \equiv x \pmod{n_i}$ for all $i$, so $y-x$ is divisible by each $n_i$, and hence by $N$. Therefore, $y \equiv x \pmod{N}$. Thus, any two solutions of the system of congruences are congruent modulo $N$, so $x$ is the unique solution modulo $N$.
	\end{proof}
\end{itemize}

\noindent \textbf{Instructions:}
\begin{enumerate}[label=\alph*.]
    \item Lorem ipsum dolor sit amet, consectetur adipiscing elit. Donec pulvinar mauris in nisi fringilla, eu dictum mi laoreet. Vestibulum et lobortis libero.
    \item Suspendisse eget massa in augue eleifend egestas. Nam pulvinar euismod enim ac tristique.
    \item Proin auctor, arcu sit amet venenatis hendrerit, nisi quam maximus justo, nec tincidunt mauris nibh nec risus. Vivamus commodo sed mauris vel dapibus.
\end{enumerate}

\noindent \textbf{Questions:}

\begin{enumerate}[label=\textbf{\arabic*.}]
    \item \textbf{Brightness (X points)} - Lorem ipsum dolor sit amet, consectetur adipiscing elit. Donec pulvinar mauris in nisi fringilla, eu dictum mi laoreet. Vestibulum et lobortis libero.
    \item \textbf{Color (X points)} - Suspendisse eget massa in augue eleifend egestas. Nam pulvinar euismod enim ac tristique.
    \item \textbf{Contrast (X points)} - Proin auctor, arcu sit amet venenatis hendrerit, nisi quam maximus justo, nec tincidunt mauris nibh nec risus. Vivamus commodo sed mauris vel dapibus.
    \item \textbf{Motion (X points)} - Lorem ipsum dolor sit amet, consectetur adipiscing elit. Donec pulvinar mauris in nisi fringilla, eu dictum mi laoreet. Vestibulum et lobortis libero.
    \item \textbf{Depth (X points)} - Suspendisse eget massa in augue eleifend egestas. Nam pulvinar euismod enim ac tristique.
\end{enumerate}

\noindent \textbf{Submission instructions:} Submit your completed assignment by emailing it to \texttt{example@email.com} by the due date.

\end{document}