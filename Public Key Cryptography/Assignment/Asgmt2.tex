\documentclass{article}

% Packages for formatting
\usepackage[margin=1in]{geometry}
\usepackage{fancyhdr}
\usepackage{enumitem}
\usepackage{graphicx}
\usepackage{kotex}
\usepackage{amsmath}
\usepackage{amsthm}
\usepackage{algorithm2e,setspace}
\usepackage{algpseudocode}
\usepackage{xcolor}
\usepackage{amssymb}

% Fonts
\usepackage[T1]{fontenc}
\usepackage[utf8]{inputenc}
\usepackage{newpxtext,newpxmath}
\usepackage{sectsty}

% Define colors
\definecolor{blue1}{HTML}{0077c2}
\definecolor{blue2}{HTML}{00a5e6}
\definecolor{blue3}{HTML}{b3e0ff}
\definecolor{blue4}{HTML}{00293c}
\definecolor{blue5}{HTML}{e6f7ff}

\definecolor{thmcolor}{RGB}{231, 76, 60}
\definecolor{defcolor}{RGB}{52, 152, 219}
\definecolor{lemcolor}{RGB}{155, 89, 182}
\definecolor{corcolor}{RGB}{46, 204, 113}
\definecolor{procolor}{RGB}{241, 196, 15}

\usepackage{color,soul}
\usepackage{soul}
\newcommand{\mathcolorbox}[2]{\colorbox{#1}{$\displaystyle #2$}}
\usepackage{cancel}
\newcommand\crossout[3][black]{\renewcommand\CancelColor{\color{#1}}\cancelto{#2}{#3}}
\newcommand\ncrossout[2][black]{\renewcommand\CancelColor{\color{#1}}\cancel{#2}}


% Chapter formatting
\definecolor{titleblue}{RGB}{0,53,128}
\usepackage{titlesec}
\titleformat{\section}
{\normalfont\sffamily\Large\bfseries\color{titleblue!100!gray}}{\thesection}{1em}{}
\titleformat{\subsection}
{\normalfont\sffamily\large\bfseries\color{titleblue!50!gray}}{\thesubsection}{1em}{}

%Tcolorbox
\usepackage[most]{tcolorbox}

%Tikzpicture
\usepackage{tikz-cd}
\usetikzlibrary{positioning}

% Header and footer formatting
\pagestyle{fancy}
\fancyhf{}
\rhead{Student ID: 20192250\quad Name: 지용현}%\rule{3cm}{0.4pt}}
\lhead{\textcolor{blue2}{\textbf{PKC Assignment \#2}}}
% Define footer
\newcommand{\footer}[1]{
	\begin{flushright}
		\vspace{2em}
		\includegraphics[width=2cm]{school_logo.jpg} \\
		\vspace{1em}
		\textcolor{blue2}{\small\textbf{#1}}
	\end{flushright}
}
%\rfoot{\large Department of Information Security, Cryptogrphy and Mathematics, Kookmin Uni.\includegraphics[height=1.5cm]{school_logo.jpg}}

\newcommand{\ie}{\textnormal{i.e.}}
\newcommand{\rsa}{\mathsf{RSA}}
\newcommand{\rsacrt}{\mathsf{RSA}\textendash\mathsf{CRT}}
\newcommand{\inv}[1]{#1^{-1}}

\usepackage{amsthm}
\newtheorem{axiom}{Axiom}[section]
\newtheorem{theorem}{Theorem}
\newtheorem*{theorem*}{Theorem}
\newtheorem{proposition}[theorem]{Proposition}
\newtheorem{corollary}{Corollary}[theorem]
\newtheorem{lemma}[theorem]{Lemma}
\newtheorem*{lemma*}{Lemma}

\theoremstyle{definition}
\newtheorem{definition}{Definition}
\newtheorem*{remark}{Remark}
\newtheorem{exercise}{Exercise}[section]

%New Command
\newcommand{\set}[1]{\left\{#1\right\}}
\newcommand{\N}{\mathbb{N}}
\newcommand{\Z}{\mathbb{Z}}
\newcommand{\Q}{\mathbb{Q}}
\newcommand{\R}{\mathbb{R}}
\newcommand{\C}{\mathbb{C}}
\newcommand{\F}{\mathbb{F}}
\newcommand{\nbhd}{\mathcal{N}}

\newcommand{\of}[1]{\left( #1 \right)} 

\begin{document}
	
	\begin{center}
		\huge\textbf{Public Key Cryptography}\\
		\vspace{0.5em}
	\end{center}
	
	%\tableofcontents
	
	\section{Review of Chinese Remainder Theorem and Fermat's Little Theorem}
	\subsection{Chinese Remainder Theorem ($\mathsf{CRT}$)}
	
	\begin{tcolorbox}[colback=white,colframe=lemcolor,arc=5pt,title={\color{white}\bf Bézout's Identity}]
		\begin{lemma*}
			$a,b\in\Z\implies\exists x,y\in\Z:\gcd\of{a,b}=ax+by$.
		\end{lemma*}
	\end{tcolorbox}
	\begin{remark}
		Let $a,b\in\Z$ and $\gcd\of{a,b}=1$. Bézout's identity guarantees $\exists x,y\in\Z:ax+by=1$ and so \[
		ax+by=1\implies ax=(-y)b+1\implies ax\equiv1\pmod{b}.
		\] That is, \[
		\mathcolorbox{-blue}{\gcd\of{a,b}=1\implies\exists x\in\Z:ax\equiv1\pmod{b}}.
		\] Similarly, if $a$ and $b$ are coprime, $b$ has a mutiplicative inverse modulo $a$.
	\end{remark}
	
	\begin{tcolorbox}[colback=white,colframe=thmcolor,arc=5pt,title={\color{white}\bf Chinese Remainder Theorem ($\mathsf{CRT}$)}]
		\begin{theorem*}
			Given a system of $k$ linear congruences:
			\begin{align*}
			x&\equiv a_1 \pmod{m_1}\\
			x&\equiv a_2 \pmod{m_2}\\
			&\vdots \\
			x&\equiv a_k \pmod{m_k}
			\end{align*} where \textcolor{magenta}{$m_1,m_2,\dots, m_k$ are pairwise coprime}. Let $M=\prod_{i=1}^km_i$. Then, the \textcolor{red}{unique} \textcolor{blue}{solution} of the system of congruences is give by \begin{align*}
			X&\equiv\sum_{i=1}^ka_iM_ib_i\pmod{M}\\
			&\equiv a_1M_1b_1+a_2M_2b_2+\cdots+a_kM_kb_k\pmod{M}.
			\end{align*} where $M_i=M/m_i$ and $ b_i\equiv\inv{M_i}\pmod{m_i}$.
		\end{theorem*}
	\end{tcolorbox}
	\begin{proof}
		(\textcolor{blue}{Existence}) Define \begin{align*}
		M&:=\prod_{i=1}^km_i=m_1m_2\cdots m_k\quad\text{and}\\
		M_i&:=\frac{M}{m_i}=m_1m_2\cdots m_{i-1}m_{i+1}\cdots m_k.
		\end{align*} By Bézout's identity, we know that \[
		\exists b_i,c_i\in\Z: M_ib_i+m_ic_i=\gcd\of{M_i,m_i}=1.
		\] because $M_i$ has not $m_i$ as factor. Then we have \[
		\begin{cases}
		M_ib_i\equiv 1\pmod{m_i}\quad\cdots\cdots(1)\\
		M_ib_i\equiv 0\pmod{m_j}\ \text{for $j\neq i$}\quad\cdots\cdots(2).
		\end{cases}
		\] Note that \begin{itemize}
			\item[(1)] $M_ib_i+m_ic_i=1\Leftrightarrow M_ib_i=(-c_i)m_i+1\Leftrightarrow M_1b_i\equiv 1\pmod{m_i}$.
			\item[(2)] \begin{align*}
			\textcolor{magenta}{\forall i,j\ \text{with}\ i\neq j:\gcd\of{m_i,m_j}=1}&\implies m_j\in\set{m_1,m_2,\cdots,m_{i-1},m_{i+1},\cdots,m_k}\\
			&\implies m_j\mid M_i\quad\because M_i=m_1m_2\cdots m_{i-1}m_{i+1}\cdots m_k\\
			&\implies m_j\mid M_i-0\\
			&\implies M_i\equiv 0\pmod{m_j}.
			\end{align*}
		\end{itemize}Then we claim that $X=\sum_{i=1}^k a_i M_ib_i$ is a solution to the system of linear congruences: \begin{align*}
		X-a_i=\of{\sum_{j=1}^k a_j M_jb_j} - a_i=\sum_{\substack{j=1\\ j\neq i}}^k a_jM_jb_j + a_iM_ib_i-a_i
		&=\sum_{j=1,j\neq i}^k a_j M_jb_j + a_i(M_ib_i - 1)\\
		&\equiv\left[\sum_{j=1,j\neq i}^k a_j\cdot\of{\crossout[red]{0}{M_jb_j }}\right]+ a_i(M_ib_i - 1)\pmod{m_i}\quad\text{by (2)}\\
		&\equiv a_i\crossout[red]{0}{(M_ib_i - 1)}\quad\pmod{m_i}\quad\text{by (1)}\\
		&\equiv 0\pmod{m_i}.
		\end{align*}
		Therefore, we have:
		$$X-a_i \equiv 0 \pmod{m_i}$$
		Hence $X$ satisfies all of the linear congruence.\\
		\\
		(\textcolor{red}{Uniqueness}) Let $X_0,X_1$ are roots of the system of linear equations. Then \begin{align*}
		X_0\equiv &a_1\equiv X_1\pmod{m_1},\\
		X_0\equiv &a_2\equiv X_1\pmod{m_2},\\
		&\vdots\\
		X_0\equiv &a_k\equiv X_1\pmod{m_k}
		\end{align*} and so \begin{align*}
		m_1&\mid X_0-X_1,\\
		m_2&\mid X_0-X_1,\\
		&\vdots\\
		m_k&\mid X_0-X_1.
		\end{align*} Hence $m_1m_2\cdots m_k\mid X_0-X_1$, \ie, \[
		X_1\equiv X_2\pmod{M=m_1m_2\cdots m_k}.
		\]
	\end{proof}
	\newpage
	
	\section{Special Case of $\mathsf{CRT}$ and $\rsa$-$\mathsf{CRT}$ Algorithm}
	
	\subsection{Two Linear Congruences for $\mathsf{CRT}$}
	
	\iffalse
	\begin{tcolorbox}[title=Chinese Remainder Theorem (CRT) - Special Case]
		\begin{theorem*}
			Given a system of $k$ linear congruences:
			\begin{align*}
			x&\equiv a_1 \pmod{m_1}\\
			x&\equiv a_2 \pmod{m_2}\\
			&\vdots \\
			x&\equiv a_k \pmod{m_k}
			\end{align*} where $m_1,m_2,\dots, m_k$ are pairwise coprime. Let $M=\prod_{i=1}^km_i$. Then, the unique solution of the system of congruences is give by \begin{align*}
			x&=\sum_{i=1}^ka_iM_ib_i\pmod{M}\\
			&=a_1M_1b_1+a_2M_2b_2+\cdots+a_kM_kb_k\pmod{M}.
			\end{align*} where $M_i=M/m_i$ and $ b_i\equiv\inv{M_i}\pmod{m_i}$.
		\end{theorem*}
	\end{tcolorbox}\fi
	\begin{tcolorbox}[title=Chinese Remainder Theorem (CRT) - Special Case]
		\begin{theorem*}
			Consider a system of two linear congruences:
			\begin{align*}
			x&\equiv a_1 \pmod{p}\\
			x&\equiv a_2 \pmod{q}
			\end{align*} where $p,q$ are coprime. Let $N=pq$. Then, the unique solution of the system of congruences is give by \begin{align*}
			\mathcolorbox{-blue}{x=a_1qq_p^{-1}+a_2pp_q^{-1}\mod{N}}
			\end{align*} where $q_p^{-1}=\inv{q}\mod{p}$ and $p_{q}^{-1}=\inv{p}\mod{q}$.
		\end{theorem*}
		\tcblower
		Recall that Bézout's identity : $
		a,b\in\mathbb{Z}\implies\exists x,y\in\mathbb{Z}:\gcd(a,b)=ax+by.
		$ Especially, \[
		\text{$p,q$ are coprime}\implies\exists x,y\in\mathbb{Z}: px+qy=1.
		\] Let $p,q$ are coprime. Then $\exists x,y\in\mathbb{Z}:px+qy=1$ and so \[
		px=(-y)q+1\quad\leadsto\quad px\equiv1\pmod{q}\quad\leadsto\quad x=p^{-1}\mod{q}.
		\] Similarly, $y=q^{-1}\mod{p}$. Thus we have $px+qy=1\leadsto pp_{q}^{-1}+qq_{p}^{-1}=1$. Consequently, \begin{align*}
		\mathcolorbox{-blue}{x=a_1qq_p^{-1}+a_2pp_q^{-1}\mod{N}} &\leftrightsquigarrow x=a_1qq_p^{-1}+a_2(1-qq_{p}^{-1})\mod{N}\\
		&\leftrightsquigarrow \mathcolorbox{-red}{x=(a_1-a_2)qq_p^{-1}+a_2\mod{N}}\\
		\end{align*}
	\end{tcolorbox}
	
	\newpage
	\subsection{$\rsa$-$\mathsf{CRT}$ Algorithm}
	
	\RestyleAlgo{ruled}
	\SetKwComment{Comment}{/* }{ */}
	\begin{algorithm}[H]
		\caption{$\mathsf{RSA}$-$\mathsf{CRT}$ Algorithm}
		\KwData{The security parameter $k$, a public key $(N,e)$, a ciphertext $\mathcal{C}$.}
		\KwResult{The plaintext message $\mathcal{M}$ corresponding to the ciphertext $\mathcal{C}$.}
		\vspace{4pt}
		\vspace{4pt}
		\SetKwFunction{KeyGen}{KeyGen}
		\SetKwProg{Fn}{Function}{:}{}
		\Comment{Key Generation}
		\Fn{\KeyGen$(1^k)$}{
			$p, q \gets$ random prime numbers of $k/2$ bits each \tcp*{Generate two primes}
			$N \gets pq$ \tcp*{Compute modulus}
			$\phi(N) \gets (p-1)(q-1)$ \tcp*{Compute Euler's phi function}
			$e \gets$ integer $e\in\of{1,\phi(N)}$ s.t. $\gcd(e, \phi(n)) = 1$ \tcp*{Choose encryption exponent}
			$d \gets$ integer $d\in\of{1,\phi(N)}$ s.t. $ed\equiv 1\pmod{\phi\of{N}}$ \tcp*{Compute decryption exponent}
			$\mathcolorbox{-green}{d_p\leftarrow d \mod p-1}$ \tcp*{Compute decryption exponent for $p$}
			$\mathcolorbox{-green}{d_q \leftarrow d \mod q-1}$ \tcp*{Compute decryption exponent for $q$}
			$q_{inv} \gets$ integer $q_{inv}\in(1, p-1)$ s.t. $qq_{inv}\equiv1 \pmod{p}$ \tcp*{Compute $q$ inverse modulo $p$}
			Set the $\rsa$ public key as $(N,e)$\;
			Set the $\rsa$ secret key as $(p,q,d_p,d_q,q_{inv})$\;
		}
		\textbf{End Function}\\
		\vspace{4pt}
		\vspace{4pt}
		\Comment{Encryption}
		\SetKwFunction{Enc}{Enc}
		\Fn{\Enc$(N,e,\mathcal{M})$}{
			$\mathcal{C}\gets\mathcal{M}^e\mod N$ \tcp*{Encrypt with $e$ and $N$}
		}
		\textbf{End Function}\\
		\vspace{4pt}
		\vspace{4pt}
		\Comment{Decryption}
		\SetKwFunction{Dec}{Dec}
		\Fn{\Dec$(\mathcal{C})$}{
			$m_1 \gets \mathcal{C}^{d_p} \mod p$ \tcp*{Decrypt with $d_p$ and $p$}
			$m_2 \gets \mathcal{C}^{d_q} \mod q$ \tcp*{Decrypt with $d_q$ and $q$}
			$t \gets q_{inv}(m_1 - m_2)$ \tcp*{Reconstruct the message using CRT}
			$m \leftarrow m_2 + qt\mod N$ \tcp*{$\mathcolorbox{-red}{m=(m_1-m_2)qq_{inv}+m_2\mod N}$}
			\KwRet{$m$}\;
		}
		\textbf{End Function}\\
		\vspace{2.5pt}
	\end{algorithm}
	\begin{proof}[Proof of Decryption]
		Note that \begin{align*}
		ed\equiv 1\pmod{(p-1)(q-1)}&\implies\exists k\in\Z:ed=1+k(p-1)(q-1),\\
		d_p=d\mod p-1&\implies\exists k_p\in\Z: d=(p-1)k_p+d_p.
		\end{align*}
		Consider $m_1:=\mathcal{C}^{d_p}\mod p$. From $\mathcal{C}=\mathcal{M}^e\mod pq$, we have \begin{align*}
		\mathcal{C}\equiv\mathcal{M}^e\pmod{pq} &\Leftrightarrow pq\mid \mathcal{C}-\mathcal{M}^e\\
		&\Leftrightarrow \mathcal{C}-\mathcal{M}^e=k_{N}\cdot pq\ \text{for some $k_{N}\in\Z$}\\
		&\Leftrightarrow \mathcal{C}-\mathcal{M}^e=\of{k_{N}p}\cdot q\\
		&\Leftrightarrow {p\mid \mathcal{C}-\mathcal{M}^e}\\
		&\Leftrightarrow \mathcolorbox{-blue}{\mathcal{C}\equiv\mathcal{M}^e\pmod{p}}.
		\end{align*} Then \begin{align*}
		m_1=\mathcal{C}^{d_p}\mod p
		&=\of{\mathcal{M}^e}^{d_p}\mod p\\
		&=\of{\mathcal{M}^e}^{d-(p-1)k_p}\mod p\\
		&=\of{\mathcolorbox{magenta}{\mathcal{M}^{ed}}}\cdot\mathcolorbox{cyan}{\mathcal{M}^{-e(p-1)k_p}}\mod p.
		\end{align*} Clearly, either $\gcd\of{\mathcal{M},p}=1$ or
		$\gcd\of{\mathcal{M},p}\neq 1$:
		\begin{itemize}
			\item[] \begin{tcolorbox}[colback=white,colframe=black,arc=5pt,title={\color{white}\bf }]
				\textbf{(Case I)} Let $\gcd\of{\mathcal{M},p}=1$. By \textcolor{red}{Fermat's little theorem}, we get \[
				\mathcolorbox{cyan}{\mathcal{M}^{-e(p-1)k_p}}=\of{\textcolor{red}{\mathcal{M}^{p-1}}}^{-ek_p}\textcolor{red}{\equiv} \textcolor{red}{1}^{-ek_p}=1\textcolor{red}{\pmod{p}}.
				\] Thus, \begin{align*}
				m_1=\mathcal{C}^{d_p}\mod p&=\of{{\mathcal{M}^{ed}}}\cdot{\mathcal{M}^{-e(p-1)k_p}}\mod p\\&=\of{{\mathcal{M}^{ed}}}\cdot 1\mod p\\
				&\equiv\mathcal{M}^{k(p-1)(q-1)+1}\pmod{p}\\
				&\equiv\of{\mathcal{M}^{p-1}}^{k(q-1)}\cdot\mathcal{M}\pmod{p}\\
				&\equiv 1^{k(q-1)}\cdot\mathcal{M}\pmod{p}\quad\text{by FLT}\\
				&\equiv\mathcal{M}\pmod{p}.
				\end{align*} That is, $m_1\equiv\mathcal{M}\pmod{p}$.
			\end{tcolorbox}
			\item[] \begin{tcolorbox}[colback=white,colframe=black,arc=5pt,title={\color{white}\bf }]
				\textbf{(Case II)} Let $\gcd\of{\mathcal{M},p}\neq 1$, \ie, $\exists l\in\Z:\mathcal{M}=pl$ since $p$ is a primes\footnote{Since $p$ is a prime, $p$ has factors $1$ and $p$ only. Then $\gcd\of{\mathcal{M},p}\neq 1$ means that $p$ is only common factor.}. Then we have \[
				\mathcal{M}=pl\implies p\mid\mathcal{M}-0\implies\mathcal{M}\equiv 0\pmod{p}.
				\] Recall that $\mathcolorbox{-blue}{\mathcal{C}\equiv\mathcal{M}^e\pmod{p}}$. Then $\mathcal{C}\equiv\mathcal{M}^e\equiv 0^e=0\pmod{p}$ and so\[
				m_1=\mathcal{C}^{d_p}\mod p=0^{d_p}\mod p= 0\mod p=0.
				\] That is, $m_1\equiv 0\pmod{p}$. Therefore \[
				\begin{cases}
				\mathcal{M}\equiv 0\pmod{p}\\
				m_1\equiv 0\pmod{p}
				\end{cases}\implies m_1\equiv\mathcal{M}\pmod{p}.
				\] 
			\end{tcolorbox}
		\end{itemize} 
		Here, we obtain $m_1\equiv\mathcal{M}\pmod{p}$. Similarly, we have $m_2\equiv\mathcal{M}\pmod{q}$. Then, the plaintext message $\mathcal{M}$ is a solution to the following system of two linear congruences: \begin{align*}
		\mathcal{M}&\equiv m_1\pmod{p},\\
		\mathcal{M}&\equiv m_2\pmod{q}.
		\end{align*}
		Hence, from Remark, we have \[
		\mathcal{M}=(m_1-m_2)qq_{inv}+m_2\mod pq.
		\]
	\end{proof}
	
	\footer{Department of Information Security, Cryptography and Mathematics, Kookmin University}
\end{document}
