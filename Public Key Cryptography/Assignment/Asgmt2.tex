\documentclass{article}

% Packages for formatting
\usepackage[margin=1in]{geometry}
\usepackage{fancyhdr}
\usepackage{enumitem}
\usepackage{graphicx}
\usepackage{kotex}
\usepackage{amsmath}
\usepackage{amsthm}
\usepackage{algorithm2e,setspace}
\usepackage{algpseudocode}
\usepackage{xcolor}
\usepackage{amssymb}

% Fonts
\usepackage[T1]{fontenc}
\usepackage[utf8]{inputenc}
\usepackage{newpxtext,newpxmath}
\usepackage{sectsty}

% Define colors
\definecolor{blue1}{HTML}{0077c2}
\definecolor{blue2}{HTML}{00a5e6}
\definecolor{blue3}{HTML}{b3e0ff}
\definecolor{blue4}{HTML}{00293c}
\definecolor{blue5}{HTML}{e6f7ff}

\definecolor{thmcolor}{RGB}{231, 76, 60}
\definecolor{defcolor}{RGB}{52, 152, 219}
\definecolor{lemcolor}{RGB}{155, 89, 182}
\definecolor{corcolor}{RGB}{46, 204, 113}
\definecolor{procolor}{RGB}{241, 196, 15}

\usepackage{color,soul}
\usepackage{soul}
\newcommand{\mathcolorbox}[2]{\colorbox{#1}{$\displaystyle #2$}}


% Chapter formatting
\definecolor{titleblue}{RGB}{0,53,128}
\usepackage{titlesec}
\titleformat{\section}
{\normalfont\sffamily\Large\bfseries\color{titleblue!100!gray}}{\thesection}{1em}{}
\titleformat{\subsection}
{\normalfont\sffamily\large\bfseries\color{titleblue!50!gray}}{\thesubsection}{1em}{}

%Tcolorbox
\usepackage[most]{tcolorbox}

%Tikzpicture
\usepackage{tikz-cd}
\usetikzlibrary{positioning}

% Header and footer formatting
\pagestyle{fancy}
\fancyhf{}
\rhead{Student ID: 20192250\quad Name: 지용현}%\rule{3cm}{0.4pt}}
\lhead{\textcolor{blue2}{\textbf{PKC Assignment \#2}}}
% Define footer
\newcommand{\footer}[1]{
\begin{flushright}
	\vspace{2em}
	\includegraphics[width=2cm]{school_logo.jpg} \\
	\vspace{1em}
	\textcolor{blue2}{\small\textbf{#1}}
\end{flushright}
}
%\rfoot{\large Department of Information Security, Cryptogrphy and Mathematics, Kookmin Uni.\includegraphics[height=1.5cm]{school_logo.jpg}}

\newcommand{\ie}{\textnormal{i.e.}}
\newcommand{\rsa}{\mathsf{RSA}}
\newcommand{\rsacrt}{\mathsf{RSA}\textendash\mathsf{CRT}}
\newcommand{\inv}[1]{#1^{-1}}

\usepackage{amsthm}
\newtheorem{axiom}{Axiom}[section]
\newtheorem{theorem}{Theorem}
\newtheorem*{theorem*}{Theorem}
\newtheorem{proposition}[theorem]{Proposition}
\newtheorem{corollary}{Corollary}[theorem]
\newtheorem{lemma}[theorem]{Lemma}
\newtheorem*{lemma*}{Lemma}

\theoremstyle{definition}
\newtheorem{definition}{Definition}
\newtheorem*{remark}{Remark}
\newtheorem{exercise}{Exercise}[section]

%New Command
\newcommand{\set}[1]{\left\{#1\right\}}
\newcommand{\N}{\mathbb{N}}
\newcommand{\Z}{\mathbb{Z}}
\newcommand{\Q}{\mathbb{Q}}
\newcommand{\R}{\mathbb{R}}
\newcommand{\C}{\mathbb{C}}
\newcommand{\F}{\mathbb{F}}
\newcommand{\nbhd}{\mathcal{N}}

\newcommand{\of}[1]{\left( #1 \right)} 

\begin{document}

\begin{center}
	\huge\textbf{Public Key Cryptography}\\
	\vspace{0.5em}
\end{center}

%\tableofcontents

\section{Review of Chinese Remainder Theorem and Fermat's Little Theorem}
\subsection{Chinese Remainder Theorem ($\mathsf{CRT}$)}

\begin{tcolorbox}[colback=white,colframe=lemcolor,arc=5pt,title={\color{white}\bf Bézout's Identity}]
	\begin{lemma*}
		$a,b\in\Z\implies\exists x,y\in\Z:\gcd\of{a,b}=ax+by$.
	\end{lemma*}
\end{tcolorbox}
\begin{remark}
	Let $a,b\in\Z$ and $\gcd\of{a,b}=1$. Bézout's identity guarantees $\exists x,y\in\Z:ax+by=1$ and so \[
	ax+by=1\implies ax=(-y)b+1\implies ax\equiv1\pmod{b}.
	\] That is, \[
	\mathcolorbox{-blue}{\gcd\of{a,b}=1\implies\exists x\in\Z:ax\equiv1\pmod{b}}.
	\] Similarly, if $a$ and $b$ are coprime, $b$ has a mutiplicative inverse modulo $a$.
\end{remark}

\begin{tcolorbox}[colback=white,colframe=thmcolor,arc=5pt,title={\color{white}\bf Chinese Remainder Theorem ($\mathsf{CRT}$)}]
	\begin{theorem*}
		Given a system of $k$ linear congruences:
		\begin{align*}
			x&\equiv a_1 \pmod{m_1}\\
			x&\equiv a_2 \pmod{m_2}\\
			&\vdots \\
			x&\equiv a_k \pmod{m_k}
		\end{align*} where \textcolor{magenta}{$m_1,m_2,\dots, m_k$ are pairwise coprime}. Let $M=\prod_{i=1}^km_i$. Then, the \textcolor{red}{unique} \textcolor{blue}{solution} of the system of congruences is give by \begin{align*}
			X&=\sum_{i=1}^ka_iM_ib_i\pmod{M}\\
			&=a_1M_1b_1+a_2M_2b_2+\cdots+a_kM_kb_k\pmod{M}.
		\end{align*} where $M_i=M/m_i$ and $ b_i\equiv\inv{M_i}\pmod{m_i}$.
	\end{theorem*}
\end{tcolorbox}
\begin{proof}
	(\textcolor{blue}{Existence}) Let $M=m_1m_2\cdots m_k$ and $M_i=M/m_i$. Then, for all $i,j\in[1,k]$ with $i\neq j$, \[
	\textcolor{magenta}{\gcd\of{m_i,m_j}=1}\implies\gcd\of{m_i, M_i}=1\implies \exists b_i: b_iM_i\equiv1\pmod{m_i}. 
	\] Then we claim that $X=\sum_{i=1}^k a_i b_i M_i$ is a solution to the system of linear congruences: \begin{align*}
		X-a_i=\of{\sum_{j=1}^k a_j b_j M_j} - a_i&=\sum_{\substack{j=1\\ j\neq i}}^k a_j b_j M_j + a_ib_iM_i-a_i
		=\sum_{j=1,j\neq i}^k a_j b_j M_j + a_i(b_iM_i - 1)
	\end{align*}
	Now, we know that $
	b_iM_i-1\equiv 0\pmod{m_i}$ since $b_iM_i\equiv 1\pmod{m_i}$. Then we get \[
	X-a_i\equiv \sum_{\substack{j=1\\ j\neq i}}^k a_j b_j M_j \pmod{m_i}.
	\]
	Now, since $\gcd\of{m_i, M_j}=1$ and $M_j$ are coprime for all $j \neq i$, we know that $a_j b_j M_j \equiv 0 \pmod{m_i}$ for all $j \neq i$. Therefore, we have:
	
	$$X-a_i \equiv 0 \pmod{m_i}$$
	
	Therefore, $X$ satisfies all of the linear congruences.
	
	Next, we will show that $X$ is unique modulo $M$. Suppose that $X'$ is another solution to the system of linear congruences. Then for any $i=1,2,\cdots, k$, we have:
	
	$$X \equiv X' \pmod{m_i} \implies m_i | (X-X')$$
	
	Since $m_i$ are pairwise coprime, we have $M=\operatorname{lcm}(m_1, m_2, \ldots, m_k)$, which implies that $M$ divides $X-X'$. Therefore, $X$ and $X'$ are congruent modulo $M$, and so $X$ is unique modulo $M$.\\
	\\
	(\textcolor{red}{Uniqueness})
\end{proof}


\subsection{Fermat's Little Theorem ($\mathsf{FLT}$)}

\newpage

\section{Special Case of $\mathsf{CRT}$ and $\rsa$-$\mathsf{CRT}$ Algorithm}

\subsection{Two Linear Congruences for $\mathsf{CRT}$}

\iffalse
\begin{tcolorbox}[title=Chinese Remainder Theorem (CRT) - Special Case]
	\begin{theorem*}
		Given a system of $k$ linear congruences:
		\begin{align*}
			x&\equiv a_1 \pmod{m_1}\\
			x&\equiv a_2 \pmod{m_2}\\
			&\vdots \\
			x&\equiv a_k \pmod{m_k}
		\end{align*} where $m_1,m_2,\dots, m_k$ are pairwise coprime. Let $M=\prod_{i=1}^km_i$. Then, the unique solution of the system of congruences is give by \begin{align*}
			x&=\sum_{i=1}^ka_iM_ib_i\pmod{M}\\
			&=a_1M_1b_1+a_2M_2b_2+\cdots+a_kM_kb_k\pmod{M}.
		\end{align*} where $M_i=M/m_i$ and $ b_i\equiv\inv{M_i}\pmod{m_i}$.
	\end{theorem*}
\end{tcolorbox}\fi
\begin{tcolorbox}[title=Chinese Remainder Theorem (CRT) - Special Case]
	\begin{theorem*}
		Consider a system of two linear congruences:
		\begin{align*}
			x&\equiv a_1 \pmod{p}\\
			x&\equiv a_2 \pmod{q}
		\end{align*} where $p,q$ are coprime. Let $N=pq$. Then, the unique solution of the system of congruences is give by \begin{align*}
			\mathcolorbox{-blue}{x=a_1qq_p^{-1}+a_2pp_q^{-1}\mod{N}}
		\end{align*} where $q_p^{-1}=\inv{q}\mod{p}$ and $p_{q}^{-1}=\inv{p}\mod{q}$.
	\end{theorem*}
	\tcblower
	Recall that Bézout's identity : $
	a,b\in\mathbb{Z}\implies\exists x,y\in\mathbb{Z}:\gcd(a,b)=ax+by.
	$ Especially, \[
	\text{$p,q$ are coprime}\implies\exists x,y\in\mathbb{Z}: px+qy=1.
	\] Let $p,q$ are coprime. Then $\exists x,y\in\mathbb{Z}:px+qy=1$ and so \[
	px=(-y)q+1\quad\leadsto\quad px\equiv1\pmod{q}\quad\leadsto\quad x=p^{-1}\mod{q}.
	\] Similarly, $y=q^{-1}\mod{p}$. Thus we have $px+qy=1\leadsto pp_{q}^{-1}+qq_{p}^{-1}=1$. Consequently, \begin{align*}
		\mathcolorbox{-blue}{x=a_1qq_p^{-1}+a_2pp_q^{-1}\mod{N}} &\leftrightsquigarrow x=a_1qq_p^{-1}+a_2(1-qq_{p}^{-1})\mod{N}\\
		&\leftrightsquigarrow \mathcolorbox{-red}{x=(a_1-a_2)qq_p^{-1}+a_2\mod{N}}\\
	\end{align*}
\end{tcolorbox}

\newpage
\subsection{$\rsa$-$\mathsf{CRT}$ Algorithm}

\RestyleAlgo{ruled}
\SetKwComment{Comment}{/* }{ */}
\begin{algorithm}[H]
	\caption{$\mathsf{RSA}$-$\mathsf{CRT}$ Algorithm}
	\KwData{The security parameter $k$, a public key $(N,e)$, a ciphertext $\mathcal{C}$.}
	\KwResult{The plaintext message $\mathcal{M}$ corresponding to the ciphertext $\mathcal{C}$.}
	\vspace{4pt}
	\vspace{4pt}
	\SetKwFunction{KeyGen}{KeyGen}
	\SetKwProg{Fn}{Function}{:}{}
	\Comment{Key Generation}
	\Fn{\KeyGen$(1^k)$}{
		$p, q \gets$ random prime numbers of $k/2$ bits each \tcp*{Generate two primes}
		$N \gets p\times q$ \tcp*{Compute modulus}
		$\phi(N) \gets (p-1)(q-1)$ \tcp*{Compute Euler's phi function}
		$e \gets$ integer $e\in\of{1,\phi(N)}$ s.t. $\gcd(e, \phi(n)) = 1$ \tcp*{Choose encryption exponent}
		$d \gets$ integer $d\in\of{1,\phi(N)}$ s.t. $ed\equiv 1\pmod{\phi\of{N}}$ \tcp*{Compute decryption exponent}
		$\mathcolorbox{-green}{d_p\leftarrow d \mod p-1}$ \tcp*{Compute decryption exponent for $p$}
		$\mathcolorbox{-green}{d_q \leftarrow d \mod q-1}$ \tcp*{Compute decryption exponent for $q$}
		$q_{inv} \gets$ integer $q_{inv}\in(1, p-1)$ s.t. $qq_{inv}\equiv1 \pmod{p}$ \tcp*{Compute $q$ inverse modulo $p$}
		Set the $\rsa$ public key as $(N,e)$\;
		Set the $\rsa$ secret key as $(p,q,d_p,d_q,q_{inv})$\;
	}
	\textbf{End Function}\\
	\vspace{4pt}
	\vspace{4pt}
	\Comment{Encryption}
	\SetKwFunction{Enc}{Enc}
	\Fn{\Enc$(N,e,\mathcal{M})$}{
		$\mathcal{C}\gets\mathcal{M}^e\mod N$ \tcp*{Encrypt with $e$ and $N$}
	}
	\textbf{End Function}\\
	\vspace{4pt}
	\vspace{4pt}
	\Comment{Decryption}
	\SetKwFunction{Dec}{Dec}
	\Fn{\Dec$(\mathcal{C})$}{
		$m_1 \gets \mathcal{C}^{d_p} \mod p$ \tcp*{Decrypt with $d_p$ and $p$}
		$m_2 \gets \mathcal{C}^{d_q} \mod q$ \tcp*{Decrypt with $d_q$ and $q$}
		$t \gets q_{inv}(m_1 - m_2)$ \tcp*{Reconstruct the message using CRT}
		$m \leftarrow m_2 + qt\mod N$ \tcp*{$\mathcolorbox{-red}{m=(m_1-m_2)qq_{inv}+m_2\mod N}$}
		\KwRet{$m$}\;
	}
	\textbf{End Function}\\
	\vspace{2.5pt}
\end{algorithm}
\begin{proof}[Proof of Decryption]
	Since $ed \equiv 1 \pmod{(p-1)(q-1)}$, \[
	\exists k\in\Z:ed = 1 + k(p-1)(q-1).
	\]
	
	Now, let's consider the following expression:
	
	$$m^{ed} = m^{1 + k(p-1)(q-1)}$$
	
	Using the binomial theorem, we can expand this expression:
	
	$$m^{1 + k(p-1)(q-1)} = m^1 \cdot (m^{(p-1)(q-1)})^k$$
	
	By Fermat's Little Theorem, we have:
	
	$$m^{p-1} \equiv 1 \pmod{p}$$
	
	Since $q-1$ is also an integer, we can raise both sides of the congruence to the power of $q-1$:
	
	$$(m^{p-1})^{q-1} \equiv 1^{q-1} \pmod{p}$$
	
	Thus, we have:
	
	$$m^{(p-1)(q-1)} \equiv 1 \pmod{p}$$
	
	So, our expanded expression becomes:
	
	$$m^{1 + k(p-1)(q-1)} \equiv m^1 \cdot 1^k \equiv m \pmod{p}$$
	
	Now, let's look at the given expressions for $c$ and $m_p$:
	
	$$c \equiv m^e \pmod{(p-1)(q-1)}$$
	$$m_p \equiv c^{d_p} \pmod{p}$$
	
	Since $d_p = d \pmod p$, we can substitute $d_p$ with $d$ in the expression for $m_p$:
	
	$$m_p \equiv c^d \pmod{p}$$
	
	From the expression for $c$, we have:
	
	$$c^d \equiv (m^e)^d \equiv m^{ed} \pmod{(p-1)(q-1)}$$
	
	However, this congruence is modulo $(p-1)(q-1)$, and we want to show the congruence modulo $p$. To do this, we can use the fact that we already proved:
	
	$$m^{ed} \equiv m \pmod{p}$$
	
	Since $m^{ed} \equiv m \pmod{p}$, and $c^d \equiv m^{ed} \pmod{(p-1)(q-1)}$, we can conclude that:
	
	$$m_p \equiv c^d \equiv m^{ed} \pmod{p}$$
	
	Thus, we have proved the desired result.
\end{proof}

We want to prove that if $\gcd(m, p) \neq 1$, then $m_p \equiv m \pmod{p}$.

Since $\gcd(m, p) \neq 1$, and $p$ is a prime, we must have $m \equiv 0 \pmod p$.

Now, let's consider the expression for $c$:

$$c \equiv m^e \pmod{(p-1)(q-1)}$$

Since $m \equiv 0 \pmod p$, we have:

$$c \equiv 0^e \pmod{(p-1)(q-1)}$$

And therefore, $c \equiv 0 \pmod p$.

Now, consider the expression for $m_p$:

$$m_p \equiv c^{d_p} \pmod p$$

Since $c \equiv 0 \pmod p$, we have:

$$m_p \equiv 0^{d_p} \pmod p$$

Now, since $d_p = d \pmod p$, we have $0 < d_p < p$. Therefore, $0^{d_p} \equiv 0 \pmod p$, and we get:

$$m_p \equiv 0 \pmod p$$

Since we have $m_p \equiv 0 \pmod p$ and $m \equiv 0 \pmod p$, it follows that:

$$m_p \equiv m \pmod p$$

Thus, we have proved the desired result.

\newpage
We want to prove that if $\gcd(m, p) \neq 1$, then $m_p \equiv m \pmod{p}$.

Since $\gcd(m, p) \neq 1$, and $p$ is a prime, we must have $m \equiv 0 \pmod p$.

Now, let's consider the expression for $c$:

$$c \equiv m^e \pmod{pq}$$

Since $m \equiv 0 \pmod p$, $m = kp$ for some integer $k$. Then, we can write:

$$c \equiv (kp)^e \pmod{pq}$$

Now we expand the expression using the binomial theorem:

$$c \equiv (kp)^e \equiv k^e \cdot p^e \pmod{pq}$$

Since $p^e$ is a multiple of $p$, it follows that $c \equiv k^e \cdot p^e \pmod{p}$. And because $m \equiv 0 \pmod{p}$, we can say that $c \equiv 0 \pmod{p}$.

Now, consider the expression for $m_p$:

$$m_p \equiv c^{d_p} \pmod p$$

Since $c \equiv 0 \pmod p$, we have:

$$m_p \equiv 0^{d_p} \pmod p$$

Now, since $d_p = d \pmod p$, we have $0 < d_p < p$. Therefore, $0^{d_p} \equiv 0 \pmod p$, and we get:

$$m_p \equiv 0 \pmod p$$

Since we have $m_p \equiv 0 \pmod p$ and $m \equiv 0 \pmod p$, it follows that:

$$m_p \equiv m \pmod p$$

Thus, we have proved the desired result.\\
---------------------------------------------\\
When we have a congruence like $c \equiv m^e \pmod{pq}$, it means that $pq$ divides $(c - m^e)$. That is, there is an integer $k$ such that:

$$c - m^e = k \cdot pq$$

Now, if we reduce this congruence modulo $p$, we have:

$$c - m^e \equiv k \cdot pq \pmod p$$

Since $p$ divides $pq$, we can say that $k \cdot pq \equiv 0 \pmod p$. Therefore, we have:

$$c - m^e \equiv 0 \pmod p$$

This can be written as:

$$c \equiv m^e \pmod p$$

So, when we reduce a congruence modulo $p$, we are essentially looking at the remainder when the number is divided by $p$. In this case, we are dividing $(c - m^e)$ by $p$, which means we are looking at the remainder of $(c - m^e)$ when divided by $p$. This gives us the value of $c$ modulo $p$.

I hope this explanation helps. Let me know if you have any further questions.
\newpage
\footer{Department of Information Security, Cryptography and Mathematics, Kookmin University}
\end{document}
