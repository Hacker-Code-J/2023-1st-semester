\documentclass{article}

% Packages for formatting
\usepackage[margin=1in]{geometry}
\usepackage{fancyhdr}
\usepackage{enumitem}
\usepackage{graphicx}
\usepackage{kotex}
\usepackage{amsmath}
\usepackage{amsthm}
\usepackage{algorithm2e,setspace}
\usepackage{algpseudocode}
\usepackage{xcolor}
\usepackage{amssymb}

% Fonts
\usepackage[T1]{fontenc}
\usepackage[utf8]{inputenc}
\usepackage{newpxtext,newpxmath}
\usepackage{sectsty}

% Define colors
\definecolor{blue1}{HTML}{0077c2}
\definecolor{blue2}{HTML}{00a5e6}
\definecolor{blue3}{HTML}{b3e0ff}
\definecolor{blue4}{HTML}{00293c}
\definecolor{blue5}{HTML}{e6f7ff}

\definecolor{thmcolor}{RGB}{231, 76, 60}
\definecolor{defcolor}{RGB}{52, 152, 219}
\definecolor{lemcolor}{RGB}{155, 89, 182}
\definecolor{corcolor}{RGB}{46, 204, 113}
\definecolor{procolor}{RGB}{241, 196, 15}

\usepackage{color,soul}
\usepackage{soul}
\newcommand{\mathcolorbox}[2]{\colorbox{#1}{$\displaystyle #2$}}
\usepackage{cancel}
\newcommand\crossout[3][black]{\renewcommand\CancelColor{\color{#1}}\cancelto{#2}{#3}}
\newcommand\ncrossout[2][black]{\renewcommand\CancelColor{\color{#1}}\cancel{#2}}

\usepackage{hyperref}
\usepackage{booktabs}

% Chapter formatting
\definecolor{titleblue}{RGB}{0,53,128}
\usepackage{titlesec}
\titleformat{\section}
{\normalfont\sffamily\Large\bfseries\color{titleblue!100!gray}}{\thesection}{1em}{}
\titleformat{\subsection}
{\normalfont\sffamily\large\bfseries\color{titleblue!50!gray}}{\thesubsection}{1em}{}

%Tcolorbox
\usepackage[most]{tcolorbox}

%Tikzpicture
\usepackage{tikz-cd}
\usetikzlibrary{positioning}
\usetikzlibrary{angles, quotes}

% Header and footer formatting
\pagestyle{fancy}
\fancyhead{}
\fancyhf{}
\rhead{Student ID: 20192250\quad Name: 지용현}%\rule{3cm}{0.4pt}}
\lhead{\textcolor{blue2}{\textbf{CA\hspace{4pt} 2023-spring-Midterm}}}
% Define footer
\newcommand{\footer}[1]{
	\begin{flushright}
		\vspace{2em}
		\includegraphics[width=2cm]{school_logo.jpg} \\
		\vspace{1em}
		\textcolor{blue2}{\small\textbf{#1}}
	\end{flushright}
}
%\rfoot{\large Department of Information Security, Cryptogrphy and Mathematics, Kookmin Uni.\includegraphics[height=1.5cm]{school_logo.jpg}}
\fancyfoot{}
\fancyfoot[C]{-\thepage-}

\newcommand{\ie}{\textnormal{i.e.}}
\newcommand{\rsa}{\mathsf{RSA}}
\newcommand{\rsacrt}{\mathsf{RSA}\textendash\mathsf{CRT}}
\newcommand{\inv}[1]{#1^{-1}}

\usepackage{amsthm}
\newtheorem{axiom}{Axiom}[section]
\newtheorem{theorem}{Theorem}
\newtheorem*{theorem*}{Theorem}
\newtheorem{proposition}[theorem]{Proposition}
\newtheorem{corollary}{Corollary}[theorem]
\newtheorem*{corollary*}{Corollary}
\newtheorem{lemma}[theorem]{Lemma}
\newtheorem*{lemma*}{Lemma}

\theoremstyle{definition}
\newtheorem{definition}{Definition}
\newtheorem*{definition*}{Definition}
\newtheorem{remark}{Remark}
\newtheorem{exercise}{Exercise}[section]

%New Command
\newcommand{\set}[1]{\left\{#1\right\}}
\newcommand{\N}{\mathbb{N}}
\newcommand{\Z}{\mathbb{Z}}
\newcommand{\Q}{\mathbb{Q}}
\newcommand{\R}{\mathbb{R}}
\newcommand{\C}{\mathbb{C}}
\newcommand{\F}{\mathbb{F}}
\newcommand{\nbhd}{\mathcal{N}}
\newcommand{\Log}{\operatorname{Log}}
\newcommand{\Arg}{\operatorname{Arg}}
\newcommand{\pv}{\operatorname{P.V.}}

\newcommand{\of}[1]{\left( #1 \right)} 
\newcommand{\abs}[1]{\left\lvert #1 \right\rvert}
\newcommand{\norm}[1]{\left\| #1 \right\|}

\newcommand{\sol}{\textcolor{magenta}{\bf Sol}}
\newcommand{\conjugate}[1]{\overline{#1}}


\renewcommand{\Re}{\operatorname{Re}}
\renewcommand{\Im}{\operatorname{Im}}

\begin{document}
	\pagenumbering{arabic}
	\begin{center}
		\huge\textbf{Complex Analysis}\\
		\vspace{0.5em}
	\end{center}
	
	\begin{enumerate}
		\item Express all complex solutions to the equation $z^6 + 1 = 0$ in the form $z = a + bi$.
		\begin{proof}[\sol]
			Let $z=re^{i\theta}$. Then \[
			z^6=r^6e^{i 6\theta}=64=2^6\implies\begin{cases}
			r = 2,\\
			6\theta = 2\pi\cdot k, \ie, \theta=\frac{k\pi}{3}\ \text{with}\ k=0,1,2,\cdots.
			\end{cases}
			\] Thus, \begin{align*}
			z_1&=2\of{\cos 0+i\sin 0}=2+0i,\\
			z_2&=2\of{\cos\of{\frac{\pi}{3}}+i\sin\of{\frac{\pi}{3}}}=1+\sqrt{3}i,\\
			z_3&=2\of{\cos\of{\frac{2\pi}{3}}+i\sin\of{\frac{\pi}{3}}}=-1+\sqrt{3}i,\\
			z_4&=2\of{\cos\pi+i\sin\pi}=-2+0i,\\
			z_5&=2\of{\cos\of{\frac{4\pi}{3}}+i\sin\of{\frac{4\pi}{3}}}=-1-\sqrt{3}i,\\
			z_6&=2\of{\cos\of{\frac{5\pi}{3}}+i\sin\of{\frac{5\pi}{3}}}=1-\sqrt{3}i,\\
			\end{align*}
			Note that: \begin{table}[ht!]
				\centering
				\begin{tabular}{c||c|c|c}
					\toprule
					Angle & $\sin(\theta)$ & $\cos(\theta)$ & $\tan(\theta)$ \\
					\midrule
					$0^\circ$ or $0$ & 0 & 1 & 0 \\
					$30^\circ$ or ${\pi}/{6}$ & ${1}/{2}$ & ${\sqrt{3}}/{2}$ & ${\sqrt{3}}/{6}$ \\
					$45^\circ$ or ${\pi}/{4}$ & ${1}/{\sqrt{2}}$ & ${1}/{\sqrt{2}}$ & 1 \\
					$60^\circ$ or ${\pi}/{3}$ & ${\sqrt{3}}/{2}$ & ${1}/{2}$ & $\sqrt{3}$ \\
					$90^\circ$ or ${\pi}/{2}$ & 1 & 0 & undefined \\
					$120^\circ$ or ${2\pi}/{3}$ & ${\sqrt{3}}/{2}$ & $-{1}/{2}$ & $-\sqrt{3}$ \\
					$135^\circ$ or ${3\pi}/{4}$ & ${1}/{\sqrt{2}}$ & $-{1}/{\sqrt{2}}$ & -1 \\
					$150^\circ$ or ${5\pi}/{6}$ & ${1}/{2}$ & $-{\sqrt{3}}/{2}$ & ${\sqrt{3}}/{3}$ \\
					$180^\circ$ or $\pi$ & 0 & -1 & 0 \\
					\bottomrule
				\end{tabular}
				%\caption{Special angles and their sine, cosine, and tangent values}
			\end{table}\\
		\end{proof}
		\vspace{8pt}
		\item Determine whether the following functions are differentiable and analytic at $z = 0$.
		\begin{enumerate}
			\item[(a)] $f\of{z}=f\of{x+iy}=\of{x+2y}+i\of{2x+y}$
			\item[(b)] $f\of{z}=\abs{z}^2+z$
			\item[(c)] $f\of{z}=f\of{x+iy}=e^x\cos y+ie^x\sin y$
			\item[(d)] $\displaystyle f\of{z}=f\of{x+iy}=\begin{cases}\displaystyle
			\frac{2x^3-3y^3}{2x^2+3y^2}+i\frac{2x^3+3y^3}{2x^2+3y^2} &:z\neq 0,\\
			0 &:z= 0.
			\end{cases}$
		\end{enumerate}
		\begin{proof}[\sol]
			\begin{enumerate}
				\item[(a)] Let $\begin{cases}
				u(x,y)=x+2y,\\
				v(x,y)=2x+y.
				\end{cases}$ Then \[
				u_x=1,\quad u_y=2,\quad v_x=2,\quad\text{and}\quad v_y=1.
				\] Thus $u_x=v_y$ but $u_y\neq -v_x$, that is, it does not hold CR-eqs.
				\vspace{4pt}
				\item[(b)] Let $z=x+iy$ then $f\of{z}=x^2+y^2+(x+iy)$. Define $\begin{cases}
				u(x,y):=x^2+x+y^2,\\
				v(x,y):=y.
				\end{cases}$ Then \[
				u_x=2x+1,\quad u_y=2y,\quad v_x=0,\quad\text{and}\quad v_y=1.
				\] And so \begin{align*}
				u_x=v_y&\implies 2x+1=1\implies 2x=0,\\
				u_y=-v_x&\implies 2y=0.
				\end{align*} Thus $f$ is differentiable at $z=0$ only. It is not holomorphic.
				\vspace{4pt}
				\item[(c)] Let $\begin{cases}
				u(x,y)=e^x\cos y,\\
				v(x,y)=e^x\sin y
				\end{cases}$ then \[
				u_x=e^x\cos y,\quad u_y=-e^x\sin y,\quad v_x=e^x\sin y,\quad\text{and}\quad v_y=e^x\cos y.
				\] Since $u_x=v_y$ and $u_y=-v_x$, a function $f$ satisfies CR-Eqs. Thus $f$ is differentiable on $\C$ and holomorphic in $\C$. And \[
				f'\of{z}=u_x+iv_x=e^x\cos y+ie^x\sin y=e^{x+iy}=e^z.
				\]
				\vspace{4pt}
				\item[(d)]
			\end{enumerate}
		\end{proof}
		\vspace{8pt}
		\item For the curve $C:z(t) = 3 + 3e^{it}$ with $0\leq t\leq\pi$, find the $\displaystyle\int_C\conjugate{z}dz$.
		\begin{proof}[\sol]
		\end{proof}
		\vspace{8pt}
		\item Let $C$ be a straight line joining $z_1=0$ to $z_2=1+i$. Find \[
		\int_C\of{3z^2+4iz}dz.
		\]
		\begin{proof}[\sol]
			content...
		\end{proof}
		\vspace{8pt}
		\item Define a principal value of complex exponent as follows: \[
		\pv z^\alpha:=\exp\of{a\Log z}.
		\] Find $\abs{\pv\of{1-i}^{1+i}}$.
		\begin{proof}[\sol]
			content...
		\end{proof}
		\vspace{8pt}
		\item Consider the complex function $h\of{z}$ defined as follows: \[
		h\of{z}=h\of{x+iy}=\of{x^3+3xy^2-3x}+i\of{y^3+3x^2y-3y}.
		\]\begin{itemize}
			\item[(a)] Show that $h\of{z}$ is differentiable at all points on $x$-axis.
			\item[(b)] Find all points where $h\of{z}$ is holomorphic.
		\end{itemize}
		\begin{proof}[\sol]
			content...
		\end{proof}
		\vspace{8pt}
		\item The curve $C$ is the unit circle $z(t) = e^(it)$ for $0\leq t\leq 2\pi$. Calculate the following integral: \begin{enumerate}
			\item[(a)] $\displaystyle\int_C\frac{2z-10}{z^2-10z}dz$.
			\item[(b)] $\displaystyle\int_C\frac{\sinh z}{z^4}dz$.
		\end{enumerate}
		\begin{proof}[\sol]
			content...
		\end{proof}
		\vspace{8pt}
		\item Let the two curves $C_R$ and $C_\rho$ in the complex plane be positively oriented semicircles, defined as follows:
		\[
		C_R: z(t) = Re^{it},\quad  C_\rho: z(t) = \rho e^{it},\quad \of{0\leq t\leq\pi}.
		\] Let the complex function $\displaystyle f(z) = \frac{z^2}{\of{z^2+1}^2}$. \begin{enumerate}
			\item[(a)] Prove the following inequality when $R=3$: \[
			\abs{\int_{C_R}f\of{z}dz}<\frac{1}{2}\pi.
			\] As the radius $R$ approaches infinity $\of{R\to\infty}$, where does the integral value converge?
			\item[(b)] Prove the following inequality when $\rho = 1/3$:
			\[
			\abs{\int_{C_\rho}f\of{z}dz}<\frac{1}{16}\pi.
			\] As the radius $\rho$ approaches zero $\of{\rho\to 0}$, where does the integral value converge?
		\end{enumerate}
		\begin{proof}[\sol]
			content...
		\end{proof}
		\vspace{8pt}
		\item Show that an entire function $f$ becomes a constant function if it satisfies the following condition: \[
		\abs{f\of{z}}\geq 1,\quad z\in\C.
		\]
		\begin{proof}[\sol]
			content...
		\end{proof}
		\vspace{8pt}
		\item For an entire function $f$ satisfying the following condition: \[
		\abs{f\of{z}}\leq 3\abs{z},\quad z\in\C,
		\] \begin{enumerate}
			\item[(a)] Show that $f^{(n)}(z) = 0$ ($n\geq 2$) for all points $z$.
			\item[(b)] Demonstrate that a function satisfying this condition is a linear polynomial, \ie, $f(z) = az + b$.
		\end{enumerate}\begin{proof}[\sol]
			content...
		\end{proof}
	\end{enumerate}
	
	
	\footer{Department of Information Security, Cryptography and Mathematics, Kookmin University}
\end{document}
