\documentclass{article}

% Packages for formatting
\usepackage[margin=1in]{geometry}
\usepackage{fancyhdr}
\usepackage{enumitem}
\usepackage{graphicx}
\usepackage{kotex}
\usepackage{amsmath}
\usepackage{amsthm}
\usepackage{algorithm2e,setspace}
\usepackage{algpseudocode}
\usepackage{xcolor}
\usepackage{amssymb}

% Fonts
\usepackage[T1]{fontenc}
\usepackage[utf8]{inputenc}
\usepackage{newpxtext,newpxmath}
\usepackage{sectsty}

% Define colors
\definecolor{blue1}{HTML}{0077c2}
\definecolor{blue2}{HTML}{00a5e6}
\definecolor{blue3}{HTML}{b3e0ff}
\definecolor{blue4}{HTML}{00293c}
\definecolor{blue5}{HTML}{e6f7ff}

\definecolor{thmcolor}{RGB}{231, 76, 60}
\definecolor{defcolor}{RGB}{52, 152, 219}
\definecolor{lemcolor}{RGB}{155, 89, 182}
\definecolor{corcolor}{RGB}{46, 204, 113}
\definecolor{procolor}{RGB}{241, 196, 15}

\usepackage{color,soul}
\usepackage{soul}
\newcommand{\mathcolorbox}[2]{\colorbox{#1}{$\displaystyle #2$}}
\usepackage{cancel}
\newcommand\crossout[3][black]{\renewcommand\CancelColor{\color{#1}}\cancelto{#2}{#3}}
\newcommand\ncrossout[2][black]{\renewcommand\CancelColor{\color{#1}}\cancel{#2}}

\usepackage{hyperref}
\usepackage{booktabs}

% Chapter formatting
\definecolor{titleblue}{RGB}{0,53,128}
\usepackage{titlesec}
\titleformat{\section}
{\normalfont\sffamily\Large\bfseries\color{titleblue!100!gray}}{\thesection}{1em}{}
\titleformat{\subsection}
{\normalfont\sffamily\large\bfseries\color{titleblue!50!gray}}{\thesubsection}{1em}{}

%Tcolorbox
\usepackage[most]{tcolorbox}

%Tikzpicture
\usepackage{tikz-cd}
\usetikzlibrary{positioning}
\usetikzlibrary{angles, quotes}

% Header and footer formatting
\pagestyle{fancy}
\fancyhead{}
\fancyhf{}
\rhead{Student ID: 20192250\quad Name: 지용현}%\rule{3cm}{0.4pt}}
\lhead{\textcolor{blue2}{\textbf{CA Assignment \#1}}}
% Define footer
\newcommand{\footer}[1]{
	\begin{flushright}
		\vspace{2em}
		\includegraphics[width=2cm]{school_logo.jpg} \\
		\vspace{1em}
		\textcolor{blue2}{\small\textbf{#1}}
	\end{flushright}
}
%\rfoot{\large Department of Information Security, Cryptogrphy and Mathematics, Kookmin Uni.\includegraphics[height=1.5cm]{school_logo.jpg}}
\fancyfoot{}
\fancyfoot[C]{-\thepage-}

\newcommand{\ie}{\textnormal{i.e.}}
\newcommand{\rsa}{\mathsf{RSA}}
\newcommand{\rsacrt}{\mathsf{RSA}\textendash\mathsf{CRT}}
\newcommand{\inv}[1]{#1^{-1}}

\usepackage{amsthm}
\newtheorem{axiom}{Axiom}[section]
\newtheorem{theorem}{Theorem}
\newtheorem*{theorem*}{Theorem}
\newtheorem{proposition}[theorem]{Proposition}
\newtheorem{corollary}{Corollary}[theorem]
\newtheorem*{corollary*}{Corollary}
\newtheorem{lemma}[theorem]{Lemma}
\newtheorem*{lemma*}{Lemma}

\theoremstyle{definition}
\newtheorem{definition}{Definition}
\newtheorem*{definition*}{Definition}
\newtheorem{remark}{Remark}
\newtheorem{exercise}{Exercise}[section]

%New Command
\newcommand{\set}[1]{\left\{#1\right\}}
\newcommand{\N}{\mathbb{N}}
\newcommand{\Z}{\mathbb{Z}}
\newcommand{\Q}{\mathbb{Q}}
\newcommand{\R}{\mathbb{R}}
\newcommand{\C}{\mathbb{C}}
\newcommand{\F}{\mathbb{F}}
\newcommand{\nbhd}{\mathcal{N}}
\newcommand{\Log}{\operatorname{Log}}
\newcommand{\Arg}{\operatorname{Arg}}
\newcommand{\pv}{\operatorname{P.V.}}

\newcommand{\of}[1]{\left( #1 \right)} 
\newcommand{\abs}[1]{\left\lvert #1 \right\rvert}
\newcommand{\norm}[1]{\left\| #1 \right\|}

\newcommand{\sol}{\textcolor{magenta}{\bf Sol}}
\newcommand{\conjugate}[1]{\overline{#1}}


\renewcommand{\Re}{\operatorname{Re}}
\renewcommand{\Im}{\operatorname{Im}}

\begin{document}
	\pagenumbering{arabic}
	\begin{center}
		\huge\textbf{Complex Analysis}\\
		\vspace{0.5em}
	\end{center}
	
	\begin{enumerate}
		\item  Express all complex solutions to the equation $z^6 + 1 = 0$ in the form $z = a + bi$.
		\begin{proof}[\sol]
			Let $z=re^{i\theta}$. Then \[
			z^6=r^6e^{i 6\theta}=64=2^6\implies\begin{cases}
			r = 2,\\
			6\theta = 2\pi\cdot k, \ie, \theta=\frac{k\pi}{3}\ \text{with}\ k=0,1,2,\cdots.
			\end{cases}
			\] Thus, \begin{align*}
			z_1&=2\of{\cos 0+i\sin 0}=2+0i,\\
			z_2&=2\of{\cos\of{\frac{\pi}{3}}+i\sin\of{\frac{\pi}{3}}}=1+\sqrt{3}i,\\
			z_3&=2\of{\cos\of{\frac{2\pi}{3}}+i\sin\of{\frac{\pi}{3}}}=-1+\sqrt{3}i,\\
			z_4&=2\of{\cos\pi+i\sin\pi}=-2+0i,\\
			z_5&=2\of{\cos\of{\frac{4\pi}{3}}+i\sin\of{\frac{4\pi}{3}}}=-1-\sqrt{3}i,\\
			z_6&=2\of{\cos\of{\frac{5\pi}{3}}+i\sin\of{\frac{5\pi}{3}}}=1-\sqrt{3}i,\\
			\end{align*}
			Note that: \begin{table}[ht!]
				\centering
				\begin{tabular}{c||c|c|c}
					\toprule
					Angle & $\sin(\theta)$ & $\cos(\theta)$ & $\tan(\theta)$ \\
					\midrule
					$0^\circ$ or $0$ & 0 & 1 & 0 \\
					$30^\circ$ or ${\pi}/{6}$ & ${1}/{2}$ & ${\sqrt{3}}/{2}$ & ${\sqrt{3}}/{6}$ \\
					$45^\circ$ or ${\pi}/{4}$ & ${1}/{\sqrt{2}}$ & ${1}/{\sqrt{2}}$ & 1 \\
					$60^\circ$ or ${\pi}/{3}$ & ${\sqrt{3}}/{2}$ & ${1}/{2}$ & $\sqrt{3}$ \\
					$90^\circ$ or ${\pi}/{2}$ & 1 & 0 & undefined \\
					$120^\circ$ or ${2\pi}/{3}$ & ${\sqrt{3}}/{2}$ & $-{1}/{2}$ & $-\sqrt{3}$ \\
					$135^\circ$ or ${3\pi}/{4}$ & ${1}/{\sqrt{2}}$ & $-{1}/{\sqrt{2}}$ & -1 \\
					$150^\circ$ or ${5\pi}/{6}$ & ${1}/{2}$ & $-{\sqrt{3}}/{2}$ & ${\sqrt{3}}/{3}$ \\
					$180^\circ$ or $\pi$ & 0 & -1 & 0 \\
					\bottomrule
				\end{tabular}
				%\caption{Special angles and their sine, cosine, and tangent values}
			\end{table}\\
		\end{proof}
	\end{enumerate}
	
	
	\footer{Department of Information Security, Cryptography and Mathematics, Kookmin University}
\end{document}
